\documentclass[letterpaper,12pt]{article}
\usepackage[utf8]{inputenc}
\usepackage[italian]{babel}
\usepackage{amsmath}
\usepackage{mathtools}
\DeclarePairedDelimiter{\abs}{\lvert}{\rvert} % per valore assoluto
\usepackage{amssymb}
\newcommand{\numberset}{\mathbb}
\newcommand{\F}{\numberset{F}}
\newcommand{\R}{\numberset{R}}

\title{Laboratorio computazionale \\[1ex] \large Appunti lezione 2}
\author{Stefano Franceschina}
\date{12/03/2025}
\begin{document}

\maketitle

\section*{Introduzione}

Come abbiamo visto, le limitazioni della rappresentazione in punto mobile si ripercuotono sulle operazioni fondamentali 
eseguite dal computer. In particolare, quando si esegue un'operazione aritmetica in macchina essa è soggetta a errori 
dovuti all'arrotondamento. Di seguito vengono illustrate alcune definizioni e concetti relativi alla rappresentazione e 
all'analisi degli errori.

\section*{Operazioni Aritmetiche Approssimate}

Si definisce la somma in macchina come:
\[
x \oplus y = \operatorname{fl}(x+y) = (x+y)\,(1+\varepsilon_1),
\]
dove $\operatorname{fl}(x+y)$ indica il risultato dell'operazione arrotondato secondo la precisione della macchina 
e $\epsilon_1 = \frac{f'-f}{1+f}$ rappresenta l'approssimazione della somma. Analogamente, per le altre operazioni si ha:
\[
x \ominus y = \operatorname{fl}(x-y) = (x-y)\,(1+\varepsilon_2),
\]
\[
x \otimes y = \operatorname{fl}(x\cdot y) = (x\cdot y)\,(1+\varepsilon_3).
\]

È evidente che queste operazioni sono approssimate e che l'insieme $\F$ dei numeri in virgola mobile non è chiuso rispetto 
a tali operazioni.

\section*{La Proprietà Associativa e l'Effetto della Cancellazione}

Uno dei problemi principali in aritmetica in punto mobile è la violazione della proprietà associativa dell'addizione. 
Consideriamo il seguente esempio, sapendo che $e = \frac{\epsilon_{mach}}{2}$
\begin{enumerate}
    \item (1.0 + e) - 1.0
    \item 1.0 + (e - 1.0)
\end{enumerate}
Ci si aspetta che entrambe le operazioni restituiscano lo stesso risultato, pari ad $e$ in particolare. Tuttavia, il 
risultato ottenuto è diverso nei due casi. Ciò avviene perchè ... (completare).

Questo fenomeno si verifica perché, nel sommare numeri con esponenti molto differenti, l'operazione richiede di allineare gli esponenti e sommare le mantisse, con conseguente perdita di significative cifre, un effetto noto come \emph{cancellazione catastrofica}.

\section*{Accuratezza e Precisione}

L'\textbf{accuratezza} indica quanto il valore calcolato si avvicini al valore reale, mentre la \textbf{precisione} 
si riferisce al numero di cifre significative della rappresentazione in macchina. Non è detto che un numero rappresentato 
con molte cifre (alta precisione) sia anche accurato, poiché un algoritmo mal progettato può produrre risultati con molte 
cifre ma errati. L'accuratezza relativa può essere espressa anche in scala logaritmica per evidenziare le differenze 
in termini di ordine di grandezza.

\section*{Algoritmi e Propagazione degli Errori}

Un algoritmo numerico può essere descritto come una funzione
\[
\tilde{f} : F \to F,
\]
cioè una funzione che prende un numero macchina e ne restituisce un altro. In pratica, un algoritmo è una composizione di funzioni:
\[
\tilde{x}_0 \xrightarrow{f_0} \tilde{x}_1 \xrightarrow{f_1} \cdots \xrightarrow{f_{N-1}} \tilde{x}_N = \tilde{y},
\]
dove $\tilde{y}$ è il risultato finale approssimato.

Siamo interessati all'errore:
\[
\Delta y = \tilde{y} - y,
\]
che, manipolando le espressioni (i conti sono riportati nelle dispense), risulta essere la somma di due contributi:
\begin{itemize}
    \item La differenza tra il valore esatto della funzione valutata nel valore reale e il valore ottenuto con la funzione troncata.
    \item L'errore introdotto dal troncamento e dall'arrotondamento durante le operazioni.
\end{itemize}

Utilizzando la relazione
\[
\tilde{x} = x\,(1+\varepsilon_x) \quad \text{con } \varepsilon_x < \varepsilon_{\text{mach}},
\]
è possibile riscrivere $\Delta y$ in funzione di $f$ e sviluppare una serie di Taylor per analizzare l'effetto degli errori.

\section*{Il Numero di Condizionamento}

Nel contesto dell'analisi degli errori, viene definita una quantità importante: il \textbf{numero di condizionamento} (\emph{condition number}). Esso fornisce informazioni su come gli errori nell'input si propaghino attraverso l'algoritmo. Ad esempio, per la funzione
\[
f(x) = x + c,
\]
il numero di condizionamento tende ad esplodere per $x=-c$, a causa della cancellazione catastrofica già menzionata.

L'altro contributo all'errore, indicato come $\Delta y_{\operatorname{fl}}$, è dovuto all'effetto del troncamento della funzione e può essere approssimato dal prodotto di $y$ per $\varepsilon_y$.

Pertanto, l'errore totale è dato dalla somma di due contributi: uno legato al numero di condizionamento (cioè, l'errore relativo all'input) e uno dovuto agli effetti di arrotondamento e troncamento nell'output.

\section*{Ulteriori Considerazioni}

È utile ricordare che il comportamento degli errori in aritmetica in virgola mobile è regolato dallo standard IEEE 754, che definisce il formato di rappresentazione, le modalità di arrotondamento e le eccezioni. La comprensione del numero di condizionamento e della propagazione degli errori è fondamentale per progettare algoritmi numerici stabili e affidabili. Algoritmi ben condizionati riescono a limitare la crescita degli errori, mentre algoritmi mal condizionati possono amplificare anche piccole imprecisioni.

\end{document}