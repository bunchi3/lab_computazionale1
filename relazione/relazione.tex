\documentclass[letterpaper,12pt]{article}
\usepackage{mathtools}
\DeclarePairedDelimiter\abs{\lvert}{\rvert}     %serve per mettere il modulo 
\usepackage{booktabs}
\usepackage{bm}
\usepackage{textcomp}
\usepackage{colortbl}
\usepackage{tabularx}
\usepackage{textcomp}
\usepackage{siunitx}
\usepackage{booktabs}
\usepackage{enumitem}
\usepackage{xcolor}
\usepackage{fancyhdr}
\usepackage{caption}
\usepackage{changepage}
\usepackage{amsmath} 
\usepackage{subcaption}
\usepackage{graphicx}
\usepackage[table]{xcolor} 
\usepackage{colortbl}
\usepackage[margin=1in,letterpaper]{geometry} % decreases margins
\usepackage{cite} % takes care of citations
\usepackage[hidelinks]{hyperref} % adds hyper links inside the generated pdf file
\usepackage{siunitx} % provides the \SI{}{} command for proper typesetting of units
% Define the colors
\definecolor{linkcolor}{RGB}{0, 102, 204}
\definecolor{citecolor}{RGB}{34, 139, 34}
\definecolor{urlcolor}{RGB}{255, 69, 0}
\definecolor{wavelength_406}{RGB}{129, 0, 204}
\definecolor{wavelength_447}{RGB}{0, 53, 255} 
\definecolor{wavelength_402}{RGB}{131, 0, 188}  
\definecolor{wavelength_501}{RGB}{0, 255, 135}
\definecolor{wavelength_440}{RGB}{0, 0, 255}   
\definecolor{wavelength_513}{RGB}{21, 255, 0}   
\definecolor{wavelength_nan}{RGB}{210,210,210} 
\definecolor{wavelength_540}{RGB}{129, 255, 0}  
\definecolor{wavelength_458}{RGB}{0, 113, 255}  
\definecolor{wavelength_568}{RGB}{219, 255, 0}
\definecolor{wavelength_587}{RGB}{255, 233, 0} 
\definecolor{wavelength_585}{RGB}{255, 239, 0} 
\definecolor{wavelength_472}{RGB}{0, 178, 255}
\definecolor{wavelength_667}{RGB}{235, 0, 0}  
\definecolor{wavelength_676}{RGB}{227, 0, 0}  
\definecolor{wavelength_640}{RGB}{255, 33, 0}  
\definecolor{wavelength_696}{RGB}{209, 0, 0}
\definecolor{wavelength_449}{RGB}{0, 65, 255}
\definecolor{wavelength_503}{RGB}{0, 255, 110}
\definecolor{wavelength_581}{RGB}{255, 252, 0}


% Setup hyperref
\hypersetup{
    colorlinks=false, % colored links
    linkcolor=linkcolor, % color for internal links
    citecolor=citecolor, % color for citations
    urlcolor=urlcolor, % color for URLs
}
\fancypagestyle{logoheader}{
    \fancyhf{}
    \fancyhead[L]{\includegraphics[width = 3cm]{logo_bicocca.png}}
    \renewcommand{\headrulewidth}{0pt}
    }
\usepackage{blindtext}
\graphicspath{{immagini/}}
%Required for inserting images
%++++++++++++++++++++++++++++++++++++++++
%Margini 


\begin{document}


\title{{\small Università degli studi Milano Bicocca - Dipartimento di Fisica}\\
	Esperimentazioni di Fisica Computazionale}
\author{S. Franceschina}
\date{\today}
\maketitle
\thispagestyle{logoheader}

%Abstract da completare
\begin{abstract} 
	\begin{adjustwidth}{-1cm}{-1cm}
	\end{adjustwidth}
\end{abstract}
\tableofcontents
\newpage

\section{Analisi dell'errore}
\subsection{Teoria}
Nella presente sezione analizziamo le due principali fonti di errore in contesti computazionali: 
\begin{enumerate}
    \item Errori di arrotondamento: dovuti alla rappresentazione di numeri reali con numero finito di digits.
    \item Errori di approssimazione: dovuti alla modalità stessa con cui affrontiamo il problema, 
          per questo motivo sono presenti anche nel caso ideale. 
\end{enumerate}

\subsection{Esercizio 1.0.1}
L'esercizio richiede di studiare $f(x)=e^x$ nell'intervallo $x\in [0,1]$, calcolando numericamente il suo sviluppo in 
serie: $g_N(x)=\sum_{n=0}^N \frac{x^n}{n}$

In particolare bisogna mostrare che
\begin{center}
\begin{equation}
    \Delta=|f(x)-g_N(x)| \approx \frac{x^{N+1}}{(N+1)!}
\end{equation}
\end{center}

\section{Sistemi lineari}
\subsection{Teoria}

\section{Radici di equazioni non lineari}
\subsection{Teoria}

\section{Interpolazioni}
\subsection{Teoria}

\section{Integrazione numerica}
\subsection{Teoria}

\section{Equazioni differenziali ordinarie}
\subsection{Teoria}

\end{document}